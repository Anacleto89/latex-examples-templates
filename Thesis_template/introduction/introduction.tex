\phantomsection
\addcontentsline{toc}{chapter}{Introduzione}
\chapter*{Introduzione}
\markboth{Introduzione}{}

\section*{Oggetto} %\label{1sec:scopo}
%
Nell'immaginario comune, il calore del terreno viene idealizzato con vulcani, sorgenti termali, fumarole ed altre manifestazioni superficiali, legate alla classica "alta entalpia" e in particolari e ristrette zone del globo sfruttate, per la sola produzione di elettricità e teleriscaldamento. Si trascurano così le ampie possibilità che la "bassa entalpia" del sottosuolo, ovunque e senza limitazioni, può offrire per la climatizzazione degli edifici, questo grazie all'aiuto delle pompe di calore.

\newacronym{cop}{$COP$}{Coefficient of Performance, letteralmente coefficiente di prestazione, si utilizza per misurare l'efficienza di una pompa di calore ed è definito come rapporto tra la quantità di calore trasportato e la quantità di energia spesa per trasportarlo; è adimensionale}
\glsadd{cop}

\newglossaryentry{entalpia}{name={entalpia}, description={Funzione di stato di un sistema che esprime la quantità di energia che esso può scambiare con l'ambiente esterno; viene usata per esprimere l'energia termica dei fluidi e fornisce un'idea del relativo valore}}
\glsadd{entalpia}

Questa impiantistica è molto conosciuta ed apprezzata all'estero, soprattutto in Svizzera, Svezia e Germania per restare nel contesto europeo e negli Stati Uniti in ambito extracomunitario. Tanto è vero che la definizione di energia geotermica fornita dalle VDI 4640 tedesche e adottata dall'EGEC, la descrive semplicemente come "l'energia immagazzinata sotto la superficie della terra solida", inglobando la doppia possibilità di utilizzazione per la produzione di calore ed elettricità.

In Europa è la Svezia a guidare il mercato col maggior numero di GSHP o impianti geotermici pro capite (circa \numprint{140000}), mentre la Svizzera ha la maggiore densità (\numprint[ogni \: km^2]{1.3}). Nel corso degli ultimi anni il mercato geotermico a bassa \gls{entalpia} ha mostrato un aumento consistente (ad esempio la Svizzera: circa il \numprint[\%]{20} all'anno per 5 anni, la Germania: oltre il \numprint[\%]{100} nel 2006) \citep{sanner}. Le ragioni di questo crescente interesse sono probabilmente imputabili alle pompe di calore, che rappresentano la tecnologia più efficiente per quanto riguarda i requisiti di energia primaria, con un risparmio del 30~$\div$~\numprint[\%]{35} rispetto alla caldaia ad olio combustibile o gas e 20~$\div$~\numprint[\%]{35} rispetto a quella a condensazione a gas. Inoltre l'uso di pompe di calore elettriche comporta l'abbattimento delle emissioni inquinanti, dal momento che queste non si verificano più nella sala tecnica, ma principalmente nelle moderne centrali dotate di costosi impianti di depurazione dei gas di scarico. Pertanto, viene assicurato un importante contributo alla riduzione delle emissioni in zone residenziali densamente popolate \citep{vdi_blatt1}. La situazione del mercato è molto positiva anche negli Stati Uniti, con più di \numprint{1000000} di pompe di calore presenti nel 2008 ed un incremento del \numprint[\%]{20} rispetto al 2007 \citep{marketusa}.

In Italia invece la diffusione su larga scala di tale tecnologia è avvenuta solo nel corso degli ultimi sei anni, diventando tumultuosa negli ultimi tre, restando però sempre confinata ad un mercato di nicchia. Gli operatori del settore hanno compreso in pieno le potenzialità economiche e ambientali, mutuando dalle esperienze all'estero. Quello che attualmente frena lo sviluppo delle PdC geotermiche può essere sintetizzato in tre mancanze: di riferimenti certi (per la Pubblica Amministrazione e per i progettisti, installatori e proprietari di immobili), di competenze tecniche e di sensibilità per le problematiche ambientali.

%%%%%%%%%%%%%%%%%%%%%%%%%%%%%%%%%%%%%%%%%%%%%%%%%%%%%%%%%%%%%%%%%%%%%%%%%
\section*{Obiettivi e contenuti}
Un argomento poco conosciuto, come la geotermia a bassa \gls{entalpia}, necessita prima di tutto di essere esaminato e descritto, in modo sintetico e chiaro, cercando anche di chiarire le molte incertezze lessicali che accompagnano la giovane materia. Si sono trattati quindi gli aspetti principali degli impianti, le proprietà termiche dei terreni e la diffusione del calore nel sottosuolo. Parte del lavoro è stata dedicata al dimensionamento iniziale delle due tipologie principali: open loop da falda sotterranea e closed loop da sonde verticali. Si è investigata la componente legata al terreno, per fornire una visione dettagliata e critica della tecnologia e per agevolare la comprensione delle problematiche ambientali connesse.


Si sono quindi descritti i molteplici aspetti che possono caratterizzare e segnare in modo indelebile la risorsa sotterranea ed i rischi collegati all'ambiente. 
In particolare:
\begin{itemize}
\item aspetti geotecnici;
\item aspetti idrogeologici;
\item aspetti costruttivi e materiali idonei.
\end{itemize}

È stata analizzata la normativa italiana vigente e gli iter delle principali province, marcando gli aspetti positivi e negativi. Si sono inoltre studiate le soluzioni previste in altri Stati, anche in questo caso ci si è soffermati sulle linee guida dei Paesi che hanno sviluppato un alta concentrazione di impianti geotermici, rimarcandone i pro ed i contro.

Si è caratterizzata poi l'idrogeologia della Provincia di Rovigo, sicuramente non esaustiva per la mancanza di uno studio sistematico e finalizzato alla tematica trattata, ma utile a far capire le criticità e le problematiche legate al sottosuolo e alle falde sotterranee presenti.

Con i risultati trovati, si sono quindi evidenziate delle semplici linee guida per la realizzazione di sistemi geotermici sostenibili.

%%%%%%%%%%%%%%%%%%%%%%%%%%%%%%%%%%%%%%%%%%%%%%%%%%%%%%%%%%%%%%%%%%%%%%%%%
\section*{Metodo}
Il criterio seguito per comporre questo testo, si è basato principalmente sullo studio e sul confronto critico della letteratura esistente. Essa comprende manuali, pubblicazioni, atti di convegno e bollettini presenti per lo più in internet e nelle sempre maggiori associazioni che raggruppano enti, progettisti ed amministrazioni, italiane e soprattutto straniere.

Da questo materiale di base, dallo studio dei due impianti geotermici più rappresentativi e dalle osservazioni prodotte e rimarcate da progettisti ed operatori del settore, si sono potuti ricavare gli impatti ambientali derivati da una cattiva progettazione e messa in opera dell'impianto, ma anche i materiali più idonei e le criticità che spesso accompagnano le prescrizioni esistenti.

Lo studio in oggetto si è articolato quindi in due fasi distinte:
\begin{enumerate}[I.]
\item bibliografica: comprendente, da un lato la raccolta dei dati e delle informazioni tecniche ed idrogeologiche, dall'altro l'indagine critica dal punto di vista ambientale e tecnico;
\item analitica: caratterizzata dalla stesura delle norme e linee guida, sulla base di documentazione tedesca, elvetica e statunitense (realtà dove la penetrazione nel mercato è ampia e ben sviluppata) adattandola alla normativa italiana vigente ed alle osservazioni poste dagli operatori del settore, frutto di convegni ed incontri pubblici.
\end{enumerate}
